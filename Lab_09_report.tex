% Digital Logic Report Template
% Created: 2020-01-10, John Miller

%==========================================================
%=========== Document Setup  ==============================

% Formatting defined by class file
\documentclass[11pt]{article}

% ---- Document formatting ----
\usepackage[margin=1in]{geometry}	% Narrower margins
\usepackage{booktabs}				% Nice formatting of tables
\usepackage{graphicx}				% Ability to include graphics

%\setlength\parindent{0pt}	% Do not indent first line of paragraphs 
\usepackage[parfill]{parskip}		% Line space b/w paragraphs
%	parfill option prevents last line of pgrph from being fully justified

% Parskip package adds too much space around titles, fix with this
\RequirePackage{titlesec}
\titlespacing\section{0pt}{8pt plus 4pt minus 2pt}{3pt plus 2pt minus 2pt}
\titlespacing\subsection{0pt}{4pt plus 4pt minus 2pt}{-2pt plus 2pt minus 2pt}
\titlespacing\subsubsection{0pt}{2pt plus 4pt minus 2pt}{-6pt plus 2pt minus 2pt}

% ---- Hyperlinks ----
\usepackage[colorlinks=true,urlcolor=blue]{hyperref}	% For URL's. Automatically links internal references.

% ---- Code listings ----
\usepackage{listings} 					% Nice code layout and inclusion
\usepackage[usenames,dvipsnames]{xcolor}	% Colors (needs to be defined before using colors)

% Define custom colors for listings
\definecolor{listinggray}{gray}{0.98}		% Listings background color
\definecolor{rulegray}{gray}{0.7}			% Listings rule/frame color

% Style for Verilog
\lstdefinestyle{Verilog}{
	language=Verilog,					% Verilog
	backgroundcolor=\color{listinggray},	% light gray background
	rulecolor=\color{blue}, 			% blue frame lines
	frame=tb,							% lines above & below
	linewidth=\columnwidth, 			% set line width
	basicstyle=\small\ttfamily,	% basic font style that is used for the code	
	breaklines=true, 					% allow breaking across columns/pages
	tabsize=3,							% set tab size
	commentstyle=\color{gray},	% comments in italic 
	stringstyle=\upshape,				% strings are printed in normal font
	showspaces=false,					% don't underscore spaces
}

% How to use: \Verilog[listing_options]{file}
\newcommand{\Verilog}[2][]{%
	\lstinputlisting[style=Verilog,#1]{#2}
}




%======================================================
%=========== Body  ====================================
\begin{document}

\title{ELC 2137 Lab 9: ALU with Input Register }
\author{Mariah Montgomery}

\maketitle


\section*{Summary}

In this lab, we created a simple arithmetic logic unit (ALU) with an input register. The ALU needed two numbers to perform a mathematical operation, so we made the numbering using switches, and we had an input register store the numbers. The circuit we built that produced the top-level module is an example of regular sequential logic because it relies on time and past outputs, whereas combinational logic does not. The first step in this experiment was to build a register. The register took input and memorized the information on the positive edge of a clock change. The memorized input was then passed to the output Q. Next, we built an ALU, which is very similar to a multiplexer. The ALU accepted the Q and a number inputted via switches. Then, based on a set parameter, the ALU performed a mathematical operation between two numbers. The ALU output was then passed to a second register that stored the input during the positive edge of a clock change. At the end of this experiment, we used our Basys3 boards to verify our ALUs.

When testing the register, we had to use the non-blocking technique because we tested sequential logic rather than combinational logic. Nonblocking does not assign elements procedurally, so it mimics hardware more accurately than blocking assignments in test benches. 



\section*{Q\&A}

\begin{table*}[ht]\centering
	\caption{\textit{Register} Expected Results Table}
	\label{ALU:tbl:register_ERT}\medskip
	\begin{tabular}{l|rrrrrrrrrrr}
		Time (ns): & 0-5 & 5-10 & 10-15 & 15-20 & 20-25 & 25-30 & 30-35 & 35-40 & 40-45 & 45-50 & 50-55 \\
		\midrule
		D (hex) & 0 & 0 	  & A & A & 3 	    & 3 	  & 0 	    & 0 & 0$\to$6 & 6 & 6 \\
		clk     & 0 & 1 	  & 0 & 1 & 0 	    & 1 	  & 0 	    & 1 & 0 	  & 1 & 0 \\
		en  	& 0 & 0 	  & 1 & 1 & 1$\to$0 & 0$\to$1 & 1$\to$0 & 0 & 0$\to$1 & 1 & 1 \\
		rst 	& 0 & 0$\to$1 & 0 & 0 & 0 		& 0 	  & 0		& 0 & 0		  & 0 & 0 \\
		\midrule
		Q (hex) & X & X$\to$0 & 0 & A & A & A & A & A & A & 6 & 6 \\
		\bottomrule
	\end{tabular}
\end{table*}

\begin{table*}[ht]\centering
	\caption{\textit{ALU} Expected Results Table}
	\label{ALU:tbl:alu_ERT}\medskip
	\begin{tabular}{l|rrrrrr}
		Time (ns): & 0-10 & 10-20 & 20-30 & 30-40 & 40-50 & 50-60 \\
		\midrule
		in0 & 0111 & 0111 & 0111 & 0111 & 0111 & 0111 \\
		in1 & 0001 & 0001 & 0001 & 0001  & 0001 & 0001 \\
		op	& 0 & 1  & 2 & 3 & 4 & 5  \\
		\midrule
		out & 1000  & 0110 & 0001 & 0111  & 0110 & 0111 \\
		\bottomrule
	\end{tabular}
\end{table*}
\pagebreak


\section*{Results}
\begin{enumerate}
	
\item Register Test Simulation

\includegraphics[width=0.8\textwidth, trim= 28.5cm 16cm 68cm 5.5cm, clip]{register_test.PNG}
\label{fig:Register Test Simulation}

\item ALU Test Simulation 

\includegraphics[width=0.8\textwidth, trim= 28.5cm 16cm 68cm 5.5cm, clip]{alu_test.PNG}
\label{fig:ALU Test Simulation}

\pagebreak

\item Top-Level Simulation 

\includegraphics[width=0.8\textwidth, trim= 28.5cm 16cm 68cm 5.5cm, clip]{top_level_test.PNG}
\label{fig:Top-Level Simulation }

\item On Board Testing Step 1 
\includegraphics[width=0.6\textwidth, angle=270]{Lab9.1.jpg} \centering
\label{fig:On Board Testing Step 1 }

\item On Board Testing Step 2 
\includegraphics[width=0.6\textwidth, angle=270]{Lab9.2.jpg} \centering
\label{fig:On Board testing Step 2}

\item On Board Testing Step 3 
\includegraphics[width=0.6\textwidth, angle=270]{Lab9.3.jpg} \centering
\label{fig:On Board testing Step 3}

\item On Board Testing Step 4 
\includegraphics[width=0.6\textwidth, angle=270]{Lab9.4.jpg} \centering
\label{fig:On Board testing Step 4}

\item On Board Testing Step 5 
\includegraphics[width=0.6\textwidth, angle=270]{Lab9.5.jpg} \centering
\label{fig:On Board testing Step 5}

\item On Board Testing Step 6 
\includegraphics[width=0.7\textwidth, angle=270]{Lab9.6.jpg} \centering
\label{fig:On Board testing Step 6}

\item On Board Testing Step 7 
\includegraphics[width=0.6\textwidth, angle=270]{Lab9.7.jpg} \centering
\label{fig:On Board testing Step 7}

\end{enumerate}

\section*{Code}

\begin{lstlisting}[style=Verilog,caption=register ,label=register.sv]
module register #(parameter N=1)
	(
		input clk,
		input rst,
		input en,
		input [N-1:0] D,
		output reg [N-1:0] Q
	);

	always @(posedge clk, posedge rst)
	begin 
		if (rst == 1)
			Q <= 0;
		else if (en == 1)
			Q <= D;
	end 
endmodule
\end{lstlisting}

\begin{lstlisting}[style=Verilog,caption=Register Test,label=register_test.sv]
module register_test();

	reg [3:0] D_t;
	reg clk, en_t, rst_t;
	wire [3:0] Q_t;

	register #(.N(4)) dut(
		.D(D_t),
		.clk(clk),
		.en(en_t),
		.rst(rst_t),
		.Q(Q_t)
	);

	always begin 
		clk = ~clk; #5;
	end

	initial begin
		clk = 0; en_t = 0; rst_t = 0; D_t = 4'h0; #7; 
		rst_t = 1; #3;
		D_t = 4'hA; en_t = 1; rst_t = 0; #10;
		D_t = 4'h3; #2;
		en_t = 0; #5;
		en_t = 1; #3;
		D_t = 4'h0; #2;
		en_t = 0; #10;
		en_t = 1; #2;
		D_t = 4'h6; #11;
		$finish;
	end 
endmodule
\end{lstlisting}

\begin{lstlisting}[style=Verilog,caption=ALU ,label=alu.sv]
module alu #(parameter N=8) 
	(
		input [N-1:0] in0,
		input [N-1:0] in1,
		input [3:0] op,
		output reg [N-1:0] out
	);

	parameter ADD=0;
	parameter SUB=1;
	parameter AND=2;
	parameter OR=3;
	parameter XOR=4;

	always @*
	begin 
		case(op)
			ADD: out = in0 + in1;
			SUB: out = in0 - in1;
			AND: out = in0 & in1;
			OR: out = in0 | in1;
			XOR: out = in0 ^ in1;
			default: out = in0;
		endcase
	end
endmodule
\end{lstlisting}

\begin{lstlisting}[style=Verilog,caption=ALU Test ,label=alu_test.sv]
module alu_test();

	reg [3:0] in0_t;
	reg [3:0] in1_t;
	reg [3:0] op_t;
	reg [3:0] out_t;

	alu #(.N(4)) dut (
		.out(out_t),
		.in0(in0_t),
		.in1(in1_t),
		.op(op_t)
	);

	initial begin
		in0_t = 4'h7; in1_t = 4'h1; op_t = 0; #10;
		in0_t = 4'h7; in1_t = 4'h1; op_t = 1; #10;
		in0_t = 4'h7; in1_t = 4'h1; op_t = 2; #10;
		in0_t = 4'h7; in1_t = 4'h1; op_t = 3; #10;
		in0_t = 4'h7; in1_t = 4'h1; op_t = 4; #10;
		in0_t = 4'h7; in1_t = 4'h1; op_t = 5; #10;
		$finish;
	end 
endmodule
\end{lstlisting}

\begin{lstlisting}[style=Verilog,caption=Top-level Module ,label=top_lab9.sv]
module top_lab9(
	input btnU,
	input btnD,
	input btnC,
	input [11:0] sw,
	input clk,
	output [15:0] led
	);

	wire [7:0] my_register1_out;
	wire [7:0] my_alu_out;
	wire [7:0] my_register2_out;

	register #(.N(8)) my_register1(
		.D(sw[7:0]),
		.clk(clk),
		.en(btnD),
		.rst(btnC),
		.Q(my_register1_out)
	);

	alu #(.N(8)) my_alu (
		.in0(sw[7:0]),
		.in1(my_register1_out),
		.op(sw[11:8]),
		.out(my_alu_out)
	);

	register #(.N(8)) my_register2(
		.D(my_alu_out),
		.clk(clk),
		.en(btnU),
		.rst(btnC),
		.Q(my_register2_out)
	);

	assign led = {my_register2_out, my_register1_out};
endmodule
\end{lstlisting}


\end{document}
